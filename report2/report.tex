\documentclass[a4paper,11pt]{article}
\usepackage[T1]{fontenc}
\usepackage[utf8]{inputenc}
\usepackage{lmodern}
\usepackage{hyperref}
\usepackage{graphicx}
\usepackage{rotating}
\usepackage{listings}
\usepackage{color}
\usepackage{listings}

\title{Algorithms, assignment - part 2 by Group 26}
\author{Arash Rouhani (rarash@student.chalmers.se) - 901117-1213\\
        Jakob Boman (bisforboman@gmail.com) - 901014-1357}

\begin{document}

\maketitle

\section{How our algorithm works}
Our two solutions are greedy in the sense that they use a \emph{strategy}
to calculate the next node in the path.

The first strategy only considers the edge-costs, whereas the other considers
only the probabilities. The first strategy picks the node which is closest
to the current node (smallest edge-cost). The other strategy picks the
remaining node which has the highest probability of containing the broken lift.


\section{Greedy Certification}

We will motivate that our algorithms are greedy. One property of greedy
is that the solution (our path) is built up step by step. It's natural
to let a \emph{step} mean a picking of the next node. So the algorithm
will always make exactly $n-1$ steps. Please note that everything mentioned
this far is general for both our algorithms, they \emph{only} differ
in their decision-making for the steps. The algorithms decision-making
is called its \emph{strategy}.

We capture the essence of building up the solution step-by-step
by defining what a strategy is and how it's used.

\subsection{Pseudocode}
\begin{lstlisting}[mathescape]
type Strategy = Graph -> Path -> Node

useGreedy :: Strategy -> Graph -> (Double, Path)
useGreedy = (pathLatency graph path, path)
  where path is the path we get by applying the strategy
        n-1 times to incrementally build the path

pathLatency was defined in the previous lab

\end{lstlisting}

The pseudocode above defines a strategy as a function
taking the graph and the path decided upon so far,
the function should return the next node to visit.

Lets look at the pseudocode definitions of our strategy

\begin{lstlisting}[mathescape]

stratCheapEdge : Strategy
stratCheapEdge =
  choose $v \in remainingNodes$ s.t. edgecost between u and v is minimal
  (remainingNodes is the set of nodes not yet visited)
  (u is the current node)

stratCheapProb : Strategy
stratCheapProb =
  choose $v \in remainingNodes$ s.t. probability(v) is maximal
  (remainingNodes is the set of nodes not yet visited)

\end{lstlisting}

Let's mention how these functions interact.
A complete 'Please take me out of here' algorithm should be of
the type $Graph -> (Double, Path)$. This is acheived by
$useGreedy(stratCheapEdge)$ and $useGreedy(stratCheapProb)$ respectively.
Note that this is an example of higher order functions.

\subsection{Complexity}
The two methods have the same complexity. Lets analyze more deeply.

\begin{lstlisting}[mathescape]
stratCheapEdge =
  choose $v \in remainingNodes$ s.t. edgecost between u and v is minimal
  (remainingNodes is the set of nodes not yet visited)
  (u is the current node)
\end{lstlisting}

This strategy is $O(n)$ in a good implementation, motivation:
"Choose the node $v \in remainingNodes$" means iterating over
the remaining nodes and then checking for "where probability(u) is maximal".
There are $O(n)$ remaining nodes, and calculating probability(u) is constant
assuming we use an array.
Calculating $remainingNodes$ can be done in $O(n)$ with arrays.

\begin{lstlisting}[mathescape]
stratCheapProb =
  choose $v \in remainingNodes$ s.t. probability(v) is maximal
  (remainingNodes is the set of nodes not yet visited)
\end{lstlisting}

This strategy is very similar to stratCheapProb. The only difference
is that we extract edge-costs instead of probabilities.
We only consider adjacent edges, which are $O(n)$.
So stratCheapEdge has also the complexity $O(n)$.


\begin{lstlisting}[mathescape]
useGreedy = (pathLatency graph path, path)
  where path is the path we get by applying the strategy
        n-1 times to incrementally build the path
\end{lstlisting}
We "build up" the path element by element. We add an
element (next node) $n-1=O(n)$ times, to calculate the
element we use the strategy, say the strategy has complexity $O(m)$.
When we have completed the path we call $pathLatency$ which in
the previous lab was $O(n)$. Therefor the complexity of
useGreedy is $O(n*m) + O(n) = O(n*m)$.

The final expressions $useGreedy(stratCheapEdge)$ and
$useGreedy(stratCheapProb)$ has the complexity $O(n^2)$
as useGreedy was $O(n*m)$ where $m$ was the complexity
of its strategy, which in both cases appeared to be
$O(n)$.

A final mention on our implementation, for convenience and
not focusing on implementation. Instead of using arrays
we use lists. This leads to the strategies
being $O(n^2)$, resulting the whole expression being $O(n^3)$.

\section{Successful and Failing instances}

We will attach the 4 instances in Fire.
See the appendix below for how to run them.

\section{Appendix}
No external resources was used.

All code, test data and more is available at the project's \href{https://github.com/bisforboman/Algorithms-TIN092}{github page}.

Testrun of the instances.

\begin{lstlisting}
> runhaskell Part2.hs < 1accept.txt
> runhaskell Part2.hs < 1fail.txt
> runhaskell Part2.hs < 2accept.txt
> runhaskell Part2.hs < 2fail.txt

\end{lstlisting}




\end{document}
