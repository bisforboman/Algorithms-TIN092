\documentclass[a4paper,11pt]{article}
\usepackage[T1]{fontenc}
\usepackage[utf8]{inputenc}
\usepackage{lmodern}
\usepackage{hyperref}
\usepackage{graphicx}
\usepackage{rotating}
\usepackage{listings}
\usepackage{color}

\title{Algorithms, assignment - part 1}
\author{Arash Rouhani, Jakob Boman}

\begin{document}

\maketitle

\section{How our algorithm works}
We generate a list of all possible paths, starting from node 1. For each path we calculate the cost using the given latency-formula, and just pick the cheapest path as our answer.

\section{Pseudocode}
Input: A complete weighted probabilistic graph G = (EC, D), where EC is a
matrix such that c = EC[u, v] is the cost for traversing from node u to v.
D is a list, so that p = D[u] is the probability of node u being
the chance of the broken lift being at node u.
A matrix is a list of lists.

Goal: To calculate the path with the lowest \emph{expected latency}.

function bestPathAndCost
input: the graph G as described above
    paths = all possible hamiltonian paths.
    pathLatencies_i = pathExpectedLatency(G, path_i) \forall. i
    return the minimum value in pathLatencies

function pathExpectedLatency
input: the graph G and the hamiltonian path P
    Using the expected latency function (EL) from the problem formulation on the input path.
    return the expected latency



\section{Appendix}
All code, images and more is available at the project's \href{https://github.com/biforboman/Algorithms-TIN092}{github page} .



\end{document}
